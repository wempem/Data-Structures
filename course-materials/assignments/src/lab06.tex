\documentclass[12pt]{article}
\usepackage{listings}
\usepackage{color}
\textwidth=7in
\textheight=9.5in
\topmargin=-1in
\headheight=0in
\headsep=.5in
\hoffset  -.85in

\definecolor{mygray}{rgb}{0.4,0.4,0.4}
\definecolor{mygreen}{rgb}{0,0.8,0.6}
\definecolor{myorange}{rgb}{1.0,0.4,0}

\lstset{
basicstyle = \ttfamily,columns=fullflexible,
commentstyle=\color{mygray},
frame=single,
numbers=left,
numbersep=5pt,
numberstyle=\tiny\color{mygray},
keywordstyle=\color{mygreen},
showspaces=false,
showstringspaces=false,
stringstyle=\color{myorange},
tabsize=2
}

\pagestyle{empty}

\renewcommand{\thefootnote}{\fnsymbol{footnote}}

\begin{document}

\begin{center}
{\bf lab 06: C++ Arrays as an Object
}
\end{center}

\setlength{\unitlength}{1in}

\begin{picture}(6,.1) 
\put(0,0) {\line(1,0){6.25}}         
\end{picture}

\renewcommand{\arraystretch}{2}
\setlength{\tabcolsep}{6pt} % General space between cols (6pt standard)
\renewcommand{\arraystretch}{.5} % General space between rows (1 standard)

\vskip.15in
\noindent\textbf{Instructions:} This lab will continue your review of arrays in C++, but now as a class.

Implement the following class
\begin{lstlisting}[language=C++]{Name=test2}
class IntArray {
    private:
    /* You fill out the private contents. */

    public:
    /* Copy array's contents to an internal array, (length = size). */
    IntArray(int *array, int size);

    /* Return the current length of the array */
    int getLength();

    /* Returns the index in the array where value is found.  
     * Return -1 if value is not present in the array.
     */
    int indexOf(int value);

    /* Removes an item at position index by shifting later elements left.
     * Returns true iff 0 <= index < size.
     */
    bool remove(int index);

    /* Returns an Array of all integers that are in common with self and ary.
     * Return an empty Array if there are no intersections.
     */
    IntArray* findIntersections(IntArray &ary);

    /* Return true if the array ary is contained sequentually in self. */
    bool isSubsequence(IntArray &ary);

    /* Delete any used memory when this variable goes out of scope. */
    ~IntArray();
}

\end{lstlisting}

\vskip.15in
\noindent\textbf{How to turn in:} \\
Turn in via GitHub.  Ensure the file(s) are in your directory and then:
\begin{itemize}
\item \$ git add $<$files$>$
\item \$ git commit 
\item \$ git push
\end{itemize}

\vskip.15in
\noindent\textbf{Due Date:}
February 1, 2017 2359

\vskip.15in
\noindent\textbf{Webhook:}
The webhook is: \\
http://coins.csuniv.edu:2234/github/build-csci-315-spring-2017.php \\
Remember, after the first push, please wait 5-10 minutes for the auto-grader to get your repository.  Then subsequent pushes should receive a grade.

\vskip.15in
\noindent\textbf{Testing:} You are in charge of testing.

\vskip.15in
\noindent\textbf{Teamwork:} No teamwork, your work must be your own.

\end{document}
