\documentclass[12pt]{article}
\usepackage{listings}
\usepackage{color}
\textwidth=7in
\textheight=9.5in
\topmargin=-1in
\headheight=0in
\headsep=.5in
\hoffset  -.85in

\definecolor{mygray}{rgb}{0.4,0.4,0.4}
\definecolor{mygreen}{rgb}{0,0.8,0.6}
\definecolor{myorange}{rgb}{1.0,0.4,0}

\lstset{
basicstyle = \ttfamily,columns=fullflexible,
commentstyle=\color{mygray},
frame=single,
numbers=left,
numbersep=5pt,
numberstyle=\tiny\color{mygray},
keywordstyle=\color{mygreen},
showspaces=false,
showstringspaces=false,
stringstyle=\color{myorange},
tabsize=2
}

\pagestyle{empty}

\renewcommand{\thefootnote}{\fnsymbol{footnote}}

\begin{document}

\begin{center}
{\bf lab 15 Binary Trees part 2}

\end{center}

\setlength{\unitlength}{1in}

\begin{picture}(6,.1) 
\put(0,0) {\line(1,0){6.25}}         
\end{picture}

\renewcommand{\arraystretch}{2}
\setlength{\tabcolsep}{6pt} % General space between cols (6pt standard)
\renewcommand{\arraystretch}{.5} % General space between rows (1 standard)

\vskip.15in
\noindent\textbf{Instructions:} This lab continues our construction of Binary Trees.  For this lab extend your previous implementation of Binary Search Tree with contains, delete, remove, existsInRange, and countInRange.

\begin{lstlisting}[language=C++]{Name=test2}
#ifndef BINARY_TREE_H
#define BINARY_TREE_H

#include <string>

template<class T>
class BinaryTreeNode {
    public:
        BinaryTreeNode<T> () {
        }
};

template<class T>
class BinaryTree {
    private:
        /* You fill in private member data. */

        /* Recommended, but not necessary helper function. */
        void put(BinaryTreeNode<T> *rover, BinaryTreeNode<T> *newNode);
        /* Recommended, but not necessary helper function. */
        std::string inorderString(BinaryTreeNode<T> *node, std::string &ret);
    public:

        /* Creates an empty binary tree. */
        BinaryTree();

        /* Does a deep copy of the tree. */
        BinaryTree(const BinaryTree<T> &tree);

        /* Add a given value to the Binary Tree. 
         * Must maintain ordering!
         */
        void put(const T &val);

        /* Returns the height of the binary tree. */
        int getHeight();

        /* Returns true if an item exists in the Binary Tree */
        bool contains(const T &val) const;

        /* Removes a specific val from the Binary Tree.
         * Returns true if the value exists (and was removed.)
         * Otherwise, returns false.
         */
        bool remove(const T &val);

        /* This method returns true iff there is a value in the tree 
         * >= min and <= max.  In other words, it returns true if there
         * is an item in the tree in the range [min, max]
         */
        bool existsInRange(T min, T max) const;

        /* This is similar but it returns the number of items in the range. */
        int countInRange(T min, T max) const;

        /* Returns a string representation of the binary Tree in order. */
        std::string inorderString();

        /* Returns a string representation of the binary Tree pre order. */
        std::string preorderString();

        /* Returns a string representation of the binary Tree pre order. */
        std::string postorderString();

        /* Does an inorder traversal of the Binary Search Tree calling
         * visit on each node.
         */
        void inorderTraversal(void (*visit) (T &item)) const;

        /* Always free memory. */
        ~BinaryTree();
};

/* Since BinaryTree is templated, we include the .cpp.
 * Templated classes are not implemented until utilized (or explicitly 
 * declared.)
 */
#include "binarytree.cpp"

#endif
\end{lstlisting}

\vskip.15in
\noindent\textbf{Write some test cases:} \\
Create some test cases, using cxxtestgen, that you believe would cover all aspects of your code.

\vskip.15in
\noindent\textbf{Memory Management:} \\
Now that are using new, we must ensure that there is a corresponding delete to free the memory.  Ensure there are no memory leaks in your code!  Please run Valgrind on your tests to ensure no memory leaks!
\vskip.15in

\vskip.15in
\noindent\textbf{How to turn in:} \\
Turn in via GitHub.  Ensure the file(s) are in your directory and then:
\begin{itemize}
\item \$ git add $<$files$>$
\item \$ git commit 
\item \$ git push
\end{itemize}

\vskip.15in
\noindent\textbf{Due Date:}
October 25, 2017 2359

\vskip.15in
\noindent\textbf{Teamwork:} No teamwork, your work must be your own.

\end{document}
