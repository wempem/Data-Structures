\documentclass[12pt]{article}
\usepackage{listings}
\usepackage{color}
\textwidth=7in
\textheight=9.5in
\topmargin=-1in
\headheight=0in
\headsep=.5in
\hoffset  -.85in

\definecolor{mygray}{rgb}{0.4,0.4,0.4}
\definecolor{mygreen}{rgb}{0,0.8,0.6}
\definecolor{myorange}{rgb}{1.0,0.4,0}

\lstset{
basicstyle = \ttfamily,columns=fullflexible,
commentstyle=\color{mygray},
frame=single,
numbers=left,
numbersep=5pt,
numberstyle=\tiny\color{mygray},
keywordstyle=\color{mygreen},
showspaces=false,
showstringspaces=false,
stringstyle=\color{myorange},
tabsize=2
}

\pagestyle{empty}

\renewcommand{\thefootnote}{\fnsymbol{footnote}}

\begin{document}

\begin{center}
{\bf lab 07: Virtual Classes and Deep Copying
}
\end{center}

\setlength{\unitlength}{1in}

\begin{picture}(6,.1) 
\put(0,0) {\line(1,0){6.25}}         
\end{picture}

\renewcommand{\arraystretch}{2}
\setlength{\tabcolsep}{6pt} % General space between cols (6pt standard)
\renewcommand{\arraystretch}{.5} % General space between rows (1 standard)

\vskip.15in
\noindent\textbf{Instructions:} In this lab, utilize a List interface to implement our Array class.  List.hpp has been provided:

\begin{lstlisting}[language=C++]{Name=test2}
#ifndef LIST_HPP
#define LIST_HPP

class List {
public:
    /* Returns the index in the array where value is found.  
     * Return -1 if value is not present in the array.
     */
    virtual int indexOf(const int value) = 0;

    /* Removes an item at position index by shifting later elements left.
     * Returns true iff 0 <= index < size.
     */
    virtual bool remove(const int index) = 0;

    /* Insert the integer val at position pos.
     * Shift all values after pos up ("to the right") by one.
     * This means the last element will be shifted out of the array 
     *    (that is fine.)
     * If pos is beyond the size of the array, increase the size of the array
     *    so val can be inserted.
     */
    virtual void insert(const int pos, const int val) = 0;

    /* Retrieves the element at position pos 
       Returns -1 if pos is invalid.*/
    virtual int get(const int pos) const = 0;

    /* Sets the element at position pos to the value val.
       Returns -1 if pos < 0.*/
    virtual int set(const int pos, const int val) = 0;

    /* Returns if the two lists contain the same elements in the
     * same order.
     */
    virtual bool equals(const List &list) = 0;
};

#endif
\end{lstlisting}

Now provide an IntArray class that implements List and utilizes deep copy constructors.  Consider the following code:

\begin{lstlisting}[language=C++]{Name=test2}
    const int ary[] = {10, 50, 34, 20};
    IntArray *ary1 = new IntArray(ary, 5);
    IntArray *ary2 = new IntArray(*ary1);
    std::cout << "(ary1 == ary2)?" << (ary1 == ary2) << "\n";
    std::cout << "ary1->equals(*ary2)?" << ary1->equals(*ary2) << "\n";
    ary2.set(2, 10);
    std::cout << "ary1->equals(*ary2)?" << ary1->equals(*ary2) << "\n";
    ary2.set(2, 34);
    std::cout << "ary1->equals(*ary2)?" << ary1->equals(*ary2) << "\n";
};
\end{lstlisting}

\vskip.15in
\noindent\textbf{Write some test cases:} \\
Create some test cases, using cxxtestgen, that you believe would cover all aspects of your code.

\vskip.15in
\noindent\textbf{Memory Management:} \\
Now that we are using new, we must ensure that there is a corresponding delete to free the memory.  Ensure there are no memory leaks in your code!

\vskip.15in
\noindent\textbf{How to turn in:} \\
Turn in via GitHub.  Ensure the file(s) are in your directory and then:
\begin{itemize}
\item \$ git add $<$files$>$
\item \$ git commit 
\item \$ git push
\end{itemize}

\vskip.15in
\noindent\textbf{Webhook:}
The webhook is: \\
http://coins.csuniv.edu:2234/github/build-csci-315-fall-2017.php \\
Remember, after the first push, please wait 5-10 minutes for the auto-grader to get your repository.  Then subsequent pushes should receive a grade.

\vskip.15in
\vskip.15in
\noindent\textbf{Due Date:}
September 18, 2017 2359

\vskip.15in
\noindent\textbf{Teamwork:} No teamwork, your work must be your own.

\end{document}
