\documentclass[12pt]{article}
\usepackage{listings}
\usepackage{color}
\textwidth=7in
\textheight=9.5in
\topmargin=-1in
\headheight=0in
\headsep=.5in
\hoffset  -.85in

\definecolor{mygray}{rgb}{0.4,0.4,0.4}
\definecolor{mygreen}{rgb}{0,0.8,0.6}
\definecolor{myorange}{rgb}{1.0,0.4,0}

\lstset{
basicstyle = \ttfamily,columns=fullflexible,
commentstyle=\color{mygray},
frame=single,
numbers=left,
numbersep=5pt,
numberstyle=\tiny\color{mygray},
keywordstyle=\color{mygreen},
showspaces=false,
showstringspaces=false,
stringstyle=\color{myorange},
tabsize=2
}

\pagestyle{empty}

\renewcommand{\thefootnote}{\fnsymbol{footnote}}

\begin{document}

\begin{center}
{\bf lab 16 Binary Heap and Heapsort}

\end{center}

\setlength{\unitlength}{1in}

\begin{picture}(6,.1) 
\put(0,0) {\line(1,0){6.25}}         
\end{picture}

\renewcommand{\arraystretch}{2}
\setlength{\tabcolsep}{6pt} % General space between cols (6pt standard)
\renewcommand{\arraystretch}{.5} % General space between rows (1 standard)

\vskip.15in
\noindent\textbf{Instructions:} This lab is a practice in constructing a basic binary heap and performing heap sort.  Implement insert, copy constructor, getHeight, getSize, contains, removeFirst, operator[], and sort.  Note: please follow the lecture as some of this will be done in class.

\begin{lstlisting}[language=C++]{Name=test2}
#ifndef HEAP_H
#define HEAP_H

#include <string>

template<class T>
class Heap {
    private:
        /* Lets fill out in class. */
    public:

        /* Creates an empty heap. */
        Heap();

        /* Does a deep copy of the array into the heap. */
        Heap(const T *array, const int size);

        /* Add a given value to the heap. 
         * Must maintain ordering!
         */
        void insert(const T &val);

        /* Returns the height of the heap. */
        int getHeight();

        /* Returns the size of the heap. */
        int getSize();

        /* Returns the index if an item if exists in the heap.
         * Otherwise return -1.
         */
        int contains(const T &val) const;

        /* Retrieves the element at position pos */
        T& operator[](const int pos);

        /* Removes and returns the first element */
        T& removeFirst();

        /* Performs a Heap Sort and returns an array of the sorted
         * elements.
         * Note: the heap is empy after the sort!
         */
        T* heapSort();

        ~Heap();
};

/* Since heap templated, we include the .cpp.
 * Templated classes are not implemented until utilized (or explicitly 
 * declared.)
 */
#include "heap.cpp"

#endif
\end{lstlisting}

\vskip.15in
\noindent\textbf{Write some test cases:} \\
Create some test cases, using cxxtestgen, that you believe would cover all aspects of your code.

\vskip.15in
\noindent\textbf{Memory Management:} \\
Now that are using new, we must ensure that there is a corresponding delete to free the memory.  Ensure there are no memory leaks in your code!  Please run Valgrind on your tests to ensure no memory leaks!
\vskip.15in

\vskip.15in
\noindent\textbf{How to turn in:} \\
Turn in via GitHub.  Ensure the file(s) are in your directory and then:
\begin{itemize}
\item \$ git add $<$files$>$
\item \$ git commit 
\item \$ git push
\end{itemize}

\vskip.15in
\noindent\textbf{Due Date:}
November 1, 2017 2359

\vskip.15in
\noindent\textbf{Teamwork:} No teamwork, your work must be your own.

\end{document}
